\chapter*{Statistics 6}

\newpage
\section{Experimental Design}

    \newpage
    \subsection{Experimental Design}
        \paragraph{Experimental error}
        Experimental error is the effect of other factors other that those controlled by the experimenter. The standard deviation is one way of measuring the experimental error, since its a measure of the spread of data. To minimize experimental error all factors that are not being investigated should be kept constant and a good experiment design should be used.

        \paragraph{Experimental design}
        The simplest experimental design is a paired comparision were experimental error is reduced bu applying the same treatment to the same subjects or in the same condition i.e. testing two diffrent diets on identical twins thus reducing experimental error due to phcological and biological effects.

        \paragraph{Randomisation}
        The purpose of randomisation is to reduce bias. Were possible the experimenter should remove all likely sources of bias by using random sampling to select an unbiased sample and random processes whenever their is a choise such as which twin is assigned which diet. 

        \paragraph{Blocking}
        Completly randomised design is the process of randomly assigning subjects to treatments and uses one-factor analysis of varience. Randomised block design is the process of spliting subjects in to simular subgroups then randomly assigning those in a subgroups to a treatment. This design is analysed using two factor analysis of variance. Randomised block design is used when another factor (a blocking factor) is thought to an effect on the results. 

        \begin{example}
            {
                In a comparision of the drying times of three different types of wood preservatives A, B and C, samples of each four diffrent woods are availible, W1, W2, W3 and W4. Three experimental designs are suggested.
                \begin{center}
                \begin{tabular}{ccc|ccc|ccc}
                & Design 1 & & & Design 2 & & & Design 3 & \\
                A & B & C & A & B & C & A & B & C \\
                \hline
                W1 & W2 & W3 & W1 & W2 & W1 & W1 & W2 & W3 \\
                W1 & W2 & W3 & W2 & W3 & W3 & W3 & W4 & W2 \\
                W1 & W2 & W3 & W3 & W1 & W2 & W4 & W1 & W4 \\
                &    &    & W4 & W4 & W4 & W2 & W3 & W1 \\
                &    &    & W1 & W2 & W3 &    &    &    \\
                \end{tabular}
                \end{center}
            }

            \begin{step}{State two dissadvantages of design 1}
            \begin{itemize}
            \item Each type of preservative is used on only on type of wood thus it would not be possible to tell if any diffrence found was due to diffrent wood or the preservative (or both).
            \item Wood 4 is not included in the test.
            \end{itemize}
            \end{step}

            \begin{step}{Write down the name of design 3}
            Randomised block design
            \end{step}

            \begin{step}{Name the tecnique used to analyse the results from design 3}
            Two factor analysis of varience
            \end{step}

            \begin{step}{State one advantage of design 3 over design 2}
            Design 2 uses some wood samples multiple times whereas design 3 uses all the samples only once. The comparision for mean drying times may be effected in design 2 by the cobination of wood samples used with each preservative.
            \end{step}

            \begin{step}{Explain how randomisation might be used in the context of design 1, 2 and 3}
            \textbf{Design 1} Allocation of wood to treatment should be carried out by a random process, ensuring the results would not be rigged by selecting a wood for the mosted favored preservative.
            
            \textbf{Design 2} A random process would decide which wood was W1 etc, thus the wood used twice would be selected randomly.

            \textbf{Design 3} Less likely to be effected by bias but the order of carrying out treatments should ideally be determined by a random process. 
            \end{step}
        \end{example}

        \paragraph{Control group}
        If a new treatment is applied to an exprimental group, a control group, which recieves no treatment is needed to measure the effects of the new treatment. This control group should as simular as possible to the exprimental group in order to minimize bias.

        \paragraph{Blind trials}
        The plecbo effect is a well known effect were patients will improve by taking a plecbo drug, a treatment which contains no active ingredient. To show that a new drug is effective many more patients must show improvment than those taking the placbo pill. In a blind trial subjects do not know if they are receiving the treatment or the placbo which help reduce bias from the plecbo effect. 
        
        \paragraph{Double blind trials} Doctors that know a patient is on the active drug may expect patients fare better which may be transmitted onto the patient. A double blind trial is were the subject and the person administrating the treatment knows wether a placebo or active drug was given.

        \paragraph{Triple blind trials} To prevent the bias in the statistical analysis it has been suggested that the statician should also not know which patients took the drug which would be a triple blind trial.

        \begin{example}
        {
            To measure the effectiveness of a drug to relive breathlessness 12 subjects, all susceptible to breathlessness, were admisnistred the drug after one attack of breathlessness and the placebo after a seperate attack. One hour after the attacks an index of breathlessness was obtained for each subject, with the following results

            \begin{center}
            \begin{tabular}{c|c|c}
            Subject & Drug & Placebo \\
            \hline
            1  & 28 & 32 \\
            2  & 31 & 33 \\
            3  & 17 & 23 \\
            4  & 18 & 26 \\
            5  & 31 & 34 \\
            6  & 12 & 17 \\
            7  & 33 & 30 \\
            8  & 18 & 19 \\
            9  & 25 & 23 \\
            10 & 19 & 21 \\
            11 & 17 & 24 \\
            12 & 16 & 49 \\
            \end{tabular}
            \end{center}
        }

        \begin{step}{Explain the role that randomisation could play in carrying out the experiment}
        \begin{itemize}
        \item While it would be impossible to obtain a random sample of all sufferers of breathlessness, a random sample of thoses that are suitable and availible could be desided by a random process.
        \item The order in which each patient takes the placbo and drug should be decided by a random process since the effect of the drug may be diffrent after the placbo (or visa-vera).
        \item Alternativly a random selected group of 6 patients would take the placbo first, and the other group would take the drug first.
        \end{itemize}
        \end{step}

        \begin{step}{Explain the meaning of a blind and of a double blind trial in the context of this expriment}
        In a blind trial the subjects would not know weather they are taking the drug or placbo. In a double blind trial, the doctors that are administrating the drug or placbo and accessing the breathlessness index would also not know weather the subject was taking the drug or placbo.
        \end{step}

        \begin{step}{Making no assumption regarding the distribution of the data, investigate the claim (using 5\% significance level) that the drug significantly reduces the breathlessness index.}
        \end{step}
        $H_0: \text{Population median diffrence} = 0$\\
        $H_1: \text{Population median diffrence} \ne 0$

        \begin{center}
        \begin{tabular}{c|c|c|c}
        Subject & Drug & Placebo & Drug - Placebo \\
        \hline
        1  & 28 & 32 & $-$ \\
        2  & 31 & 33 & $-$ \\
        3  & 17 & 23 & $-$ \\
        4  & 18 & 26 & $-$ \\
        5  & 31 & 34 & $-$ \\
        6  & 12 & 17 & $-$ \\
        7  & 33 & 30 & $+$ \\
        8  & 18 & 19 & $-$ \\
        9  & 25 & 23 & $+$ \\
        10 & 19 & 21 & $-$ \\
        11 & 17 & 24 & $-$ \\
        12 & 16 & 49 & $-$ \\
        \end{tabular}
        \end{center}

        $$
        10^- / 2^+
        $$

        \begin{align*}
        X &\sim B(12, 0.5) \\
        P(X \leq 2) &= P(X < 3) \\
        &= 0.0193
        \end{align*}
        
        Since $0.0193 < 0.05$ reject $H_0$ at the 5\% significance level the drug reduceses breathlessness.

        \end{example}
    
    \newpage
    \subsection{Analysis of paired comparisions}
        Assuming their are two normal populations from which paired samples of size $n$ are taken, with means $\mu_1$ and $\mu_2$, then the diffrence between the pairs will also be normal distributed. Thus a test for $\mu_1 = \mu_2$ is equivilent to $\mu_d = 0$.

        $$
        \bar{D} \sim N\left(\mu_d, \frac{\sigma_d^2}{n}\right)
        $$

        $$
        Z = \dfrac{\bar{D} - \mu_d}{\dfrac{\sigma_d}{\sqrt{n}}} \sim N(0, 1)
        $$
        Let $\bar{D}$ and $S_d^2$ denote the mean and variance of the a sample of n diffrences. Thus
        $$
        \frac{\bar{D} - \mu_d}{\dfrac{S_d}{\sqrt{n}}} \sim t_{n-1}
        $$

        \begin{example}
        {
            A school mathamatics teachers wants to test the effect of a new educational computer package. She selects pairs of students of equal ability and randomly selects one from each pair to join the experimental group (with the others making the control group). The results of a later assesment are as follows:

            \begin{center}
            \begin{tabular}{c|c|c}
            Pair & Control & Experimental \\
            \hline
            1  & 72 & 75 \\
            2  & 82 & 79 \\
            3  & 93 & 84 \\
            4  & 65 & 71 \\
            5  & 76 & 82 \\
            6  & 89 & 91 \\
            7  & 81 & 85 \\
            8  & 58 & 68 \\
            9  & 95 & 90 \\
            10 & 91 & 92 \\
            \end{tabular}
            \end{center}

            Assuming the diffrence in marks are normally distributed, investigate the claim that the computer package inproves students understanding using a 5\% significance level.
        }

        \begin{step}{Hypothesis}
        $H_0: \mu_d = 0$\\
        $H_0: \mu_d > 0$
        \end{step}

        \begin{step}{Test statistic}
        \begin{align*}
        \bar{d} &= 1.5\\
        s_d &= 5.720\\
        t &= \frac{1.5 - 0}{\dfrac{5.720}{\sqrt{10}}}\\
        &= 0.8293
        \end{align*}
        \end{step}

        \begin{step}{Critical value}
        $$t_{10 - 1}(0.95) = 1.833$$
        \end{step}

        \begin{step}{Conclusion}
        $0.8293 < 1.833$ so accept $H_0$, their is no evidence at the 5\% significance level that the computer package improves the students results.
        \end{step}

        \end{example}
\newpage
\section{Analysis of Variance}

\newpage
\section{Statistical Process Control}

\newpage
\section{Acceptance Sampling}
