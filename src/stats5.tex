\chapter*{Statistics 5}

\newpage
\section{Continuous Probability Distributions}

    \newpage
    \subsection{Rectanglular Distribution}
        The rectangular distribution is a random variable that takes any value in given range within an interval, the pdf of which is:
        $$
        f(x) = 
        \begin{cases}
        \frac{1}{b - a} & a \leq x \leq b\\
        0 & \text{otherwise}
        \end{cases}
        $$
        The mean and varience are given by:
        $$\operatorname{E}(x) = \frac{b + a}{2}$$
        $$\operatorname{Var}(x) = \frac{(b - a)^2}{12}$$
        And the cumulative distribution function can be found by integrating $f(x)$ between x and the lower bound.

        \begin{example}
        {
            Given $
            f(x) = 
            \begin{cases}
                \frac{1}{6} & -2 \leq x \leq 4\\
                0 & \text{otherwise}
            \end{cases}
            $
        }

        \begin{step}{Find the mean}
        \begin{align*}
        E(x) &= \frac{4 + -2}{2} \\
             &= 1
        \end{align*}
        \end{step}

        \begin{step}{Find the varience}
        \begin{align*}
        Var(x) &= \frac{(4 - -2)^2}{12} \\
               &= 3
        \end{align*}
        \end{step}

        \begin{step}{Find the value of $P(|X| < 1)$}
        \begin{align*}
        P(|X| < 1) &= P(-1 < X < 1) \\
                   &= 2 \times \frac{1}{6} \\
                   &= \frac{1}{3}
        \end{align*}
        \end{step}

        \end{example}

    \newpage
    \subsection{Exponential Distribution}
        The exponential distribution is the interval between the successive events of a Poisson distribution. Its density function is as follows:
        $$
        f(x)
        \begin{cases}
        \lambda e^{-\lambda x} & x > 0\\
        0 & \text{otherwise}\\
        \end{cases}
        $$
        where lambda is the parameter of the exponential distribution. The mean and varience are given by:
        $$\operatorname{E}(x) = \frac{1}{\lambda}$$
        $$\operatorname{Var}(x) = \frac{1}{\lambda}$$
        If you integrate between $0$ and $x$, you can find the cumulative distribution function:
        \begin{align*}
        \int^{x}_{0}{\lambda e^{-\lambda x}}
        &= [-e^{-\lambda x}]^x_0\\
        &= [-e^{-\lambda x} -- e^{0}]^x_0\\
        &= [-e^{-\lambda x} + 1]\\
        &= 1 - e^{-\lambda x}
        \end{align*}

        \begin{example}
        {
        The number of cars passing a point per minute is 0.8. Find the probability that the interval between two cars is longer than 2 minuets.
        }

        \begin{step}{Define the density function}
        $$
        F(x) = 1 - e^{-0.8x}
        $$
        \end{step}

        \begin{step}{Find the probabilty}
        \begin{align*}
        F(\infty) - F(2) 
        &= (1 - e^{-0.8 \times \infty}) - (1 - e^{-0.8 \times 2})\\
        &= (1 - 0) - (1 - e^{-1.6})\\
        &= 0.202
        \end{align*}
        \end{step}

        \end{example}

\newpage
\section{Estimation}

    \newpage
    \subsection{Confindence interval for varience}

\newpage
\section{Hypothesis Testing}

    \newpage
    \subsection{Test for varience}
    
    \newpage
    \subsection{F-distribution}
    
    \newpage
    \subsection{Test for varience equality}
    
    \newpage
    \subsection{Test for mean equailty}
    
    \newpage
    \subsection{Goodness of fit test}